\section{Conclusion}

The program was capable of re-discovering the minimal forbidden minors for $F(1)$, $F(2)$, $F(3)$, as well as identifying a number of elements of $F(4)$ and $F(5)$. All the generated minimal forbidden minors are publicly available online, and can be subject to verification through a panoply of means, given the panoply of formats published. On the procedure itself, the obvious limitation of the approach were algorithmic. There is a formidable number potential candidate graphs and it is challenging to efficiently enumerate or sample in such large spaces. Nevertheless, it is possible to identify a number of areas that can be dived upon for improvement in future research. On one hand, it would be useful to tune the graph generation procedure, so that graphs that cannot possibly be minimal forbidden minors (e.g. graphs with nodes of degree 2, graphs with multiple biconnected components, etc.) are never generated. Essentially, more extensive and focused pruning mechanisms. Relatedly, one could explore sampling via the topic of k-trees, which are edge-maximal graphs for a given treewidth. Above all, more ambitious algorithm engineering is bound to achieve the goal of finding close to 90 to 100 forbidden minors in $F(4)$, and likely for higher values of $k$ too. It would help if the constituent parts of the pipeline were to run more quickly, or simply more efficiently. A collaborative approach, which looks to farm out graphs (in the style of SETI@Home \cite{seti2002}) out of a fully enumerated online database would be an interesting way forward. Perhaps using the 'nauty' swift and light C++ generation tool 'geng' (using the .g6 file format), converting those graphs into a format this program can work with and constantly updating the results is one way of guaranteeing exhaustiveness. The hosting of such a service would then be available for any researchers willing to spend CPU resources on the search for minimal forbidden minors. Going forward, a public ever-evolving forbidden minor database and online checking tool would be an exciting outcome of the concepts tackled in this thesis.