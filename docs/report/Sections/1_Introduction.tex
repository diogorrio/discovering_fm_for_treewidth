\section{Introduction}

\subsection{Contextualization}
Folklore, unproven theories, talk about the existence of around 90 to 100 minimal forbidden minors in F(4). This is, however, unknown, and simply intuition-based by some researchers of the topic.

What is known is that a finite limit does exist. As a starting point, the sets F(1), F(2), and F(3) are known and fully proven to be fully enumerated. Nothing of the sort composes the case for k $>$ 3.

Therefore, the foundation of this thesis is held.

The importance of this task has to do with the fact that analyzing the treewidth of a graph may allow us to make fair conclusions regarding the computational time of NP-hard problems.

\subsection{Thesis Outline}
This thesis looks to explore the nature of the algorithms that allow the generation of graphs as well as discern relevant pruning techniques that may significantly reduce the time spent searching the space.

Different tools are used, not always in the same environment/language, so to speed up the entire process of filling F(4) and F(5). For this reason, conversions are regularly conducted to allow compatibility between functions and their results.

A MySQL database structure is also implemented to provide the possibility of sharing the results obtained by this research. 

All these algorithms, methods of conversion and sharing are thus described throughout the paper.

\subsection{Problem Statement }
[Look into what it could be]
\subsubsection{Research Questions}
Here are some of the questions the research aims to address:

1. How can we algorithmically efficiently discover the list of 90 to 100 minors for F(4)?

2. How can the list of minors we find be communicated to the community, in a verifiable way?

3. Is there any evidence that this list for F(4) is (or is not) exhaustive?

4. Can the same algorithmic techniques be used to produce a (non-exhaustive) list for F(5)?

