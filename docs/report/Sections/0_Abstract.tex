\date{}

\section*{Abstract}

This project is aimed to develop an algorithm or a series of them capable of discovering around 90 to 100 minimal forbidden minors in F(4). Sequentially, the same methods will be applied to F(5), to scout whether an initial list can be found there.

These sets are characterized by the concept of treewidth.

Given an undirected graph $G=(V, E)$, the treewidth of a graph is a parameter that describes 'how far that graph is from being a tree'. This is quite important in algorithmic graph theory, as it allows NP-hard problems to be solved in polynomial, and sometimes even in linear, time. That is, of course, if the treewidth of the graphs in question is bounded (or small). This makes understanding whether a graph has or does not have high treewidth fundamental.

Graphs have a finite set of forbidden minors \cite{robertson2004wagners}. A minor is a subgraph that is obtainable by deleting vertices and edges and contracting edges. If a graph has treewidth at most k, a set F(k) characterizes said graph. Therefore, to check whether any graph has treewidth at most k, it suffices to guarantee that it has none of the graphs in F(k) as a minor. 

In addition, given that we only know that F(k) is finite, it would be an interesting research outcome if the 90-to-100-sized set F(4) ended up being exhaustive. 

The algorithmic challenge comprises generating all the valid graphs and then searching the space looking for graphs with treewidth k+1 (or higher), for F(k) - but so that when a minor is taken, the treewidth drops to k (or smaller). This takes applying efficient search-space pruning techniques, pre-processing, exact treewidth solvers, and connectivity/isomorphism checks.

[Missing results, their interpretations and conclusions]


\vspace{0.5cm}

% Keywords (may be sorted in alphabetical order)
\noindent
\textbf{Keywords:}
Forbidden Minors, Treewidth, NP-hardness, F(k), Pruning, Connectivity, Isomorphism

\vspace{1.0cm}

\section*{Acknowledgment}
I would like to use this section to thank my supervisor Steven Kelk, for the guidance throughout the project - as well as for suggesting the topic.


\newpage
\renewcommand*\contentsname{Table of Contents}
\tableofcontents

\newpage
