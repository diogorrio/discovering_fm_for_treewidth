\date{}
\section*{Abstract}
The aim of this thesis is to explore the efficiency of algorithms dedicated to finding minimal forbidden minors in $F(4)$, where $F(k)$ is the finite set of graphs with the following property: a graph $G$ has treewidth at most $k$ if and only if none of the graphs in $F(k)$ are a minor of $G$. The treewidth of a graph is a fundamental characteristic that describes its resemblance to a tree. This is crucial in graph theory, as it enables polynomial time solutions for NP-hard problems, for graphs with bounded treewidth. Methods of communicating the discovered forbidden minors to the community in a verifiable manner are developed, as only up the set $F(3)$ are they fully known. Finally, the research aims to analyze the exhaustiveness of the obtained list of forbidden minors in $F(4)$, as well as examine the feasibility of extending the algorithms used in $F(4)$ in generating a non-exhaustive list for $F(5)$.\par To achieve these goals, the study employs graph generation and exploration techniques, search-space pruning methods, and connectivity and isomorphism checks. Different approaches are evaluated, and a MySQL database structure is implemented to facilitate the sharing of any research outcomes.\par It defines an endeavor to advance the field of algorithmic graph theory and its possible applications in solving complex computational problems.\par
\vspace{0.5cm}
% Keywords (may be sorted in alphabetical order)
\noindent
\textbf{Keywords:}
Forbidden Minors, Treewidth, NP-hardness, Pruning, Connectivity, Isomorphism
\section*{Acknowledgment}
I would like to use this section to thank my supervisor Steven Kelk, for the guidance throughout the project - as well as for suggesting the topic.
\renewcommand*
\contentsname{Table of Contents}
\tableofcontents
